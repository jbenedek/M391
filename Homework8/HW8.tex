\documentclass{article}
\usepackage{amsmath}
\usepackage{amsthm}
\usepackage{graphicx}
\usepackage{tikz}
\usepackage{amsfonts}

% Options for amsthm
\newtheorem*{thm}{Theorem}
\newtheorem*{ex}{Exercise}
\newtheorem*{cor}{Corollary}
\newtheorem{lem}{Lemma}

\newenvironment{solution}
  {\begin{proof}[Solution]}
  {\renewcommand{\qedsymbol}{}\end{proof}}

\title{Homework $8$}
\author{Jeremy Benedek}
\date{March 20, 2016}

\begin{document}
\maketitle

\begin{ex}[3.14]
	Show that $\oplus$ is well-defined on $\mathbb{Z}_n$.
\end{ex}
\begin{solution}
	$2\oplus3=5$ and $12\oplus 15 = 29 = 5 + 2(12) = 5$
	So the representation of the number did not change the 
	result, so this is well-defined.
\end{solution}

\begin{ex}[A.1]
	Simply each of the following expressions modulo the specified
	number. \begin{itemize}
	  \item $(a+b)^3 \bmod 3$
	  \item $103-28+5(6) \bmod 2$
	  \item $3(4) \bmod 6$
	  \item $1586749 \bmod 10$
	\end{itemize}
\end{ex}
\begin{solution}
	\begin{itemize}
	  \item $a+b)^3 \bmod 3$ \\ $n(a+b)^3 = 3$
	  \item $103-28+30 = 75+30 = 105$ \\ $105 \bmod 2 = 1$
	  \item $12 \bmod 6 = 0$
	  \item $1586749 \bmod 10 = 9$
	\end{itemize}
\end{solution}

\begin{ex}[3.16]
	Looking at the Cayley tables for $C_5$ and $\mathbb{Z}_5$, do you
	notice any features of the rows and columns that could be
	generalized to all Cayley tables? Make a conjecture end prove
	it.
\end{ex}
\begin{solution}
	I  notice that lines of diagonal have the same number. This makes
	sense since the rows and columns of a Cayley table are in
	sequential order. With the operation of addition, when you
	move up one column, you also move right one row (making the 
	diagonal), so you start with $a+b$ and move a up ($a-1$) and
	move b right ($b+1$). So $a+b = (a-1) + (b+1)$. These
	numbers are equivalent, so the numbers along a diagonal are equivalent.
	\qed
\end{solution}

\begin{ex}[3.18]
	Consider a square in the plane. How many distinct symmetries does
	it have? Give each symmetry a concise, meaningful label. For each
	pair of symmetries, S and T, compose them in both orders. Record
	this in a Cayley table. 
\end{ex}
\begin{solution}
	There are 8 symmetries: $R_0, R_{90}, R_{180}, R_{270}, D_1, D_2,
	M_1, M_2$, where $R_n$ is a rotation of n degrees. $D_1$ and $D_2$
	are both flips along a diagonal line, and $M_1$ and $M_2$ are flips
	are the midpoints (horizontal and vertical) of the sides.

	Composing these symmetries in different orders will result 
	in different transformation. $R_{90} \circ D_2 \neq D_2 \circ R_{90}$

\end{solution}

\begin{thm}[3.19]
	The symmetries of the square in the plane with composition
	form a group.
\end{thm}
\begin{proof}
	This is a group because it is Associative, since function
	composition is, the identity element is $R_0$, there exists
	an inverse for all transformation, and the operation is closed
	and well-defined. 
\end{proof}

\begin{ex}[3.20]
	For each natural number $n$, how many elements does $D_n$ have? 
	Justify.
\end{ex}
\begin{solution}
	$D_n = 2n$, this comes from there being two ways to organize the
	vertices, clockwise and counter-clockwise. For each group, there
	are n organizations. So it follows that the total is $2n$. 
\end{solution}

\begin{ex}[3.22]
	For each of the following groups, find all subgroups. Argue that
	your list is complete.

	\begin{enumerate}
	  \item $(D_4, \circ)$
	  \item $(\mathbb{Z}, +)$
	  \item $(C_n, \oplus_n)$
	\end{enumerate}
\end{ex}
\begin{solution}
	\begin{enumerate}
	  \item The subset including an element, its inverse, and the identity
	    will be a sub group. $R_0$ is also a subgroup on its own.
	    All rotations are also a subgroup on its own. The super group
	    is also a subgroup. Each flip individually with $R_0$ will
	    be subgroups. A complete list will come from fully analyzing 
	    the Cayley table and finding groups of elements that work.
	  \item There are many subgroups, $(n\mathbb{Z}, +)$ are many subgroups. 
	    This list is complete since multiples of intergers will result
	    in a subgroup of this supergroup.
	  \item There are many subgroups formed by changing the $n$. 
	    This is a complete list since this cyclic group
	    can including many groups with a different n focus. 
	\end{enumerate}


\end{solution}`

\end{document}
