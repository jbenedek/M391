\documentclass{article}
\usepackage{amsmath}
\usepackage{amsthm}
\usepackage{graphicx}
\usepackage{tikz}
\usepackage{amssymb}


% Options for amsthm
\newtheorem*{thm}{Theorem}
\newtheorem*{ex}{Exercise}
\newtheorem*{cor}{Corollary}
\newtheorem{lem}{Lemma}

\newenvironment{solution}
  {\begin{proof}[Solution]}
  {\renewcommand{\qedsymbol}{}\end{proof}}

\title{Reflection $3$}
\author{Jeremy Benedek}
\date{March 11, 2016}

\begin{document}
\maketitle

\begin{description}
	\item[Part 1: Exam Preparation] \hfill \\
	 	To prepare for the next proof based exam, I am going to 
		carefully review previous proofs that were completed for homework
		and focus on the feedback of the grader. This past exam had a few proofs we have seen before, so buy learning what I did wrong,
		I could do well on those proofs if I were to see them again. Also, this will help me review some common tricks or themes
		that occurred in proofs we did, as well as to discover some common mistakes I have made.
		The second thing I would do to study would be to create proof outlines, this is a great way to 
		prepare for an exam since you are thinking about specific proof structures that you are likely to see on the exam. Finally, another
		method to prepare for an exam is to take a claim we have already proven, and prove it using a different method. This allows you
		to try out different methods and gain some tricks and new perspectives that may help you when solving a different and difficult proof. 

	\item[Part 2: Groups vs Graphs] \hfil \\
		Graphs and groups are similar in that they both involve sets, a graph is a pair of sets, and a group is a special set with an operator,
		that is how these two things are the same, so the proofs for them, since they are sets, will follow the proof of a set; we can
		you use the same proof techniques, contrapositive, contradiction, etc. 
		Both structures (groups and graphs), are made up of many elements, such as individual edges and vertices in a graph, or individual
		elements or numbers in a group. So a proof for them requires you to prove the claim for every element, so you must prove it 
		generically. 
		They differ in that Group theory has a much more prevalent base in algebra and algebraic manipulation. 
		Additionally, for me, graph theory proofs seem more abstract as you are trying to prove something about a graph, without depending on 
		the pictorial representation, where as group theory proofs seem more concrete and straight to the point (so far, at least), since
		you are proving something about a group using algebra. 

	\item[Part 3: Three Polished Proofs] \hfil \\
		\begin{thm}[2.82]
	  	Any planar graph with no loops is 6-colorable.
	      \end{thm}
	      \begin{proof}
			Following Theorem 2.63, any planar graph with no loops has a vertex of degree at most 5. Using Theorem 2.79,  graph $G$ is $m+1$ colorable, with $m$ being the biggest degree of a vertex.
			Since 5 is the largest degree of a vertex in a planar graph, it is 5+1, or 6-colorable.
	      \end{proof}
			This proof has a result that is important, since the result of the proof is so overarching. The result of this proof
			relates to all planar graphs, so since we proved this, we now know something about whole grouping of graphs.

			\hfil \\

		\begin{thm}[2.80]
		  	Consider a graph, $G$, that is built from a subgraph, $H$, by adding one new vertex, $v$, and new edges that connect the new vertex to vertices in $H$.
			If the subgraph $H$ has a 5-coloring such that the new vertex, $v$, is 
			not adjacent to vertices of all five colors, then $G$ is 5-colorable.
		\end{thm}
		\begin{proof}
			Consider vertex $v$, which is the vertex added to graph $H$ (along with edges) that creates $G$. If $H$ is 5-colorable, and $v$ is not adjacent to all five colors, then if you assign $v$ to be a color
			that is not the color of an adjacent vertex, you will maintain the colorability of the graph. So, $G$ is still 5-colorable.
		 \end{proof}
			This proof has a interesting argument, since you start by knowing something about one graph, and the claim also claims
			that you can build up that graph by adding a vertex, and get a similar result. More specifically, this argument
			is interesting since by changing the graph you start with, you are not changing the result (if you follow the restrictions
			that were put on in the claim of this proof). 


		\begin{thm}[2.78]
		  	For any natural number, $n$, let $G = (V,E)$ be a graph with $|V| \leq n$ that has no loops. Then $G$ is n-colorable. 
	      \end{thm}
	      \begin{proof}
			Take graph $G$ and assign each vertex a distinct color. Since $G$ has no loops, no 2 adjacent vertices have the same color.
			So each vertex has a different color, and no vertex is adjacent to itself. 
			$G$ is now n-colorable, if $n$ is the number of colors used and $|V| \leq n$. 
	      \end{proof}
			This exposition has three characteristics of good style, it is short and concise, it explains the notation and variables used,
			and it restates the claim, including the restrictions that the claims placed upon it.

\end{description}
\end{document}
