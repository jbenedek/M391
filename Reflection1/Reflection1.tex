\documentclass{article}
\usepackage{amsmath}
\usepackage{amsthm}
\usepackage{graphicx}
\usepackage{tikz}

% Options for amsthm
\newtheorem*{thm}{Theorem}
\newtheorem*{ex}{Exercise}
\newtheorem*{cor}{Corollary}
\newtheorem{lem}{Lemma}

\newenvironment{solution}
  {\begin{proof}[Solution]}
  {\renewcommand{\qedsymbol}{}\end{proof}}

\title{Reflection $1$}
\author{Jeremy Benedek}
\date{January 29, 2016}

\begin{document}
\maketitle

\begin{description}
	\item[Part 1: Proof Construction] \hfill \\
		When I am stuck on a math problem or proof, I like to work through examples of the argument to fully convince myself that
		what I am proving is true. If that doesn't work, then I like to think about the goals for both the entire argument 
		and whatever step I am stuck on. Once I have a goal to
		work through, the problem is more manageable. If that doesn't work, then I think about what is stumping me,
		and how can I get over the hurdle. Finally, if none of that works, I double check my work looking for errors
		that would interfere with  me solving it, and if I don't see any, I restart the problem.  
	\item[Part 2: Presentations] \hfil \\
		To increase the clarity of a presentation, a presenter should write big and legibly so all students can see it. Additionally, 
		a presenter should step back from the board occasionally to allow students to read what they just wrote, and also to explain 
		the step they just wrote.  Finally, a presenter should be willing to answer questions from the audience.  
	
	\item[Part 3: Three Polished Proofs] \hfil \\
		This is why this is interesting... show proof

		\begin{thm}[number]
    			Theorem statement here.
		\end{thm}

		\begin{proof}
    			Write your proof here.
		\end{proof}


\end{description}
\end{document}
