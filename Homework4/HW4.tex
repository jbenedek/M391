\documentclass{article}
\usepackage{amsmath}
\usepackage{amsthm}
\usepackage{graphicx}
\usepackage{tikz}

% Options for amsthm
\newtheorem*{thm}{Theorem}
\newtheorem*{ex}{Exercise}
\newtheorem*{cor}{Corollary}
\newtheorem{lem}{Lemma}

\newenvironment{solution}
  {\begin{proof}[Solution]}
  {\renewcommand{\qedsymbol}{}\end{proof}}

\title{Homework $4$}
\author{Jeremy Benedek}
\date{Feburary 7, 2016}

\begin{document}
\maketitle

\begin{thm}[2.35]
	A graph $G$ has an Euler path if and only if $G$ is connected and has zero or two vertices of odd degree and all other vertices have even,
	positive degree.
\end{thm}
\begin{proof}
	Assume $G$ is connected, so there is a walk between all vertices. You can have either 0 or 2 odd degree vertices. If you have 0 
	odd degree vertices, then you have an Euler circuit, which is an Euler path. Otherwise, you start at one of the odd degree vertices, walk
	through the graph. You will then end at the other vertex of odd degree.
\end{proof}

\begin{ex}[2.36]
	Give an example of an graph with an Euler path but not an Euler Circuit. What must be true of any such example?
\end{ex}
\begin{solution}
	$G = (V,E)$, $V=\{A,B,C,D\}$, $E=\{ \{A,B\}, \{B,C\}, \{C,D\} \}$ \\ There must be 2 vertices of odd degree, for this to be true. 
\end{solution}

\begin{ex}[2.40]
	By drawing a few examples, explore the relationship between the number of degree one vertices of a tree and other features of the tree.
	Make a conjecture and prove it.
\end{ex}
\begin{solution}
	After drawing a few examples, a relations ship I saw was that the number of leaves must be greater than or equal than 2, while
	also being less than or equal to the total number of edges. \\ 
	\begin{proof} 	Consider the smallest tree possible, a tree with 2 vertices connected by 1 edge. Both vertices have a degree of 1, so they
		are both leaves. So the least amount of leaves possible is 2. Next, consider some other tree. Besides the smallest tree,
		each edge can contribute at most one leave. Therefore the maximum amount of leaves in a tree is equal to the amount of edges
	      . \end{proof} 
\end{solution}

\begin{thm}[2.42]
	If $v$ and $w$ are distinct vertices of a tree $G$, then there is a unique walk with no repeated edges in $G$ from $v$ to $w$.
\end{thm}
\begin{proof}
	A tree is a connected graph with no circuit. If you remove an edge, then the graph is no longer connected since there are no
	circuits, so there is no other way for a walk to exist. So you have a unique walk between $v$ and $w$. Using Theorem 2.22,
	since we have a walk, we have a walk with no repeated edges between $v$ and $w$.
	
\end{proof}

\begin{cor}[2.43]
	Suppose $G = (V,E)$ is a tree and $e$ is an edge in $E$. Then the subgraph $G\prime = (V,E \setminus \{e\})$ is not connected.   
\end{cor}
\begin{proof}
	Assume towards contradiction $G^\prime$ is connected. Since there are no repeated edges (Thm. 2.42) and also no circuits, 
	removing an edge will create a disconnect in the walk, making the graph not connected. Therefore, $G^\prime$ is not connected. 
\end{proof}

\begin{thm}[2.44]
	Let $G$ be a connected graph. Then there is a subtree, $T$ of $G$ that contains every vertex of $G$.
\end{thm}
\begin{proof}
	Start with graph $G$ and make a sub graph that contains no circuits. You now have a connected graph with no circuits, a tree. Since
	$G$ was connected, you can make a subgraph with all vertices in $G$. This subgraph now contains all vertices of $G$, and is a tree.
\end{proof}

\begin{thm}[2.45]
	A subtree $T$ in a connected graph $G$ is a maximal tree if and only if $T$ contains every vertex of $G$. 
\end{thm}
\begin{proof}
	Assume $T$ is a maximal tree. Then $T$ contains all non-circuit causing edges of $G$. Since it contains all edges that don't complete a circuit
	all vertices of $G$ must be in $T$. Now assume $T$ contains every vertex of $G$. This is a requirement for $T$ to be a maximal tree, the other
	being that any edge of $G$ not in $T$ causes $T$ to not be a tree. So, $T$ can only be a maximal tree if it contains every vertex of $G$. 
\end{proof}

\begin{ex}[2.47]
	Draw graphs of $K_3$, $K_4$, $K_5$, $K_{2,3}$, $K_{3,3}$, $K_{2,4}$. What appear to be planar graphs? Are any of them familiar?
\end{ex}
\begin{solution}
	 $K_3$, $K_4$, $K_{2,3}$, $K_{2,4}$ are planar. $K_5$ is the 5 station model and $K_{3,3}$ is the Utility problem. We've seen these 2 problems
	 previously. See next page for the graphs of this exercise. 
\end{solution}


\end{document}
