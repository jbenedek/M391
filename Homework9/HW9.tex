\documentclass{article}
\usepackage{amsmath}
\usepackage{amsthm}
\usepackage{graphicx}
\usepackage{tikz}
\usepackage{amsfonts}
% Options for amsthm
\newtheorem*{thm}{Theorem}
\newtheorem*{ex}{Exercise}
\newtheorem*{cor}{Corollary}
\newtheorem{lem}{Lemma}

\newenvironment{solution}
  {\begin{proof}[Solution]}
  {\renewcommand{\qedsymbol}{}\end{proof}}

\title{Homework $9$}
\author{Jeremy Benedek}
\date{March 27, 2016}

\begin{document}
\maketitle

\begin{thm}[3.24]
	Let $G$ be a group with identity element $e$. Then $\{e\}$ is a subgroup of $G$. 
\end{thm}
\begin{proof}
	$\{e\}$ is a group since it has an identity element, $e$, and all members have an inverse, in this case with only one element, $\{e\}$ has
	an inverse of $\{e\}$. The associativity and closed requirements of a group are inherited from the $G$, so it follows that $\{e\}$ also
	have these characteristics. This will be well defined since it is the identity element by itself, so regardless of the operation, $e*e$ will
	have a result always in $\{e\}$. Therefore, this is a group, and since all members are in $G$, this is a subgroup of $G$.
\end{proof}

\begin{thm}[3.25]
	Let $G$ be a group. Then $G$ is a subgroup of $G$. 
\end{thm}
\begin{proof}
	Since $G$ is a group, $G$ is closed, well-defined, associative, has an identity element, and all members have an inverse. So it follows
	that $G$ is a group. Since $G$ is the same group as $G$, all members of $G$ exist in $G$, so $G$ is a subgroup of $G$.
\end{proof}

\begin{ex}[3.26]
	Let $G$ be a group, $g\in G$, and $n,m\in\mathbb{Z}$. Then \begin{enumerate}
	  \item $g^ng^m = g^{n+m}$
	  \item $(g^n)^{-1}$
	\end{enumerate}

\end{ex}
\begin{solution}
	\begin{enumerate}
	  \item $g^n$ is $g$ stared to itself $n$ times. Like wise $g^m$ is $g$ stared to itself $m$ times, where star is the operation. So if you
	    combine the two, you get $g$ stared to itself $n$ times and then $m$ times, so $n+m$ times. $g$ stared to itself $n+m$ times is
	    the same as $g^{n+m}$. 
	  \item By definition, $g^{-n}$ is the binary operation applied to the inverse of $g$ $n$ times. So, $g^{-1}$ starred $n$ times. $g^n$ is $g$ 
	    starred to itself $n$ times. The inverse of this would be the inverse of the individual elements starred together $n$ times, following
	    the socks and shoes lemma. So the inverse of $g^n$ would be the inverse of $g$ starred together $n$ times. Which is the same as $g^{-n}$.
	    So these two have the same result. 
	\end{enumerate}
\end{solution}

\begin{thm}[3.27]
	Let $G$ be a group and $g$ be an element of $G$. Then $\langle g \rangle$ is a subgroup of $G$. 
\end{thm}
\begin{proof}
	All members of $\langle g \rangle$ are members of $G$ since $\langle g \rangle$ is formed by using the members of $G$. $\langle g \rangle$
	is also a group since it is associative and well-defined since that is inherited from the bigger group. The identity element 
	is present in this subgroup since $e = g^n * g^{-n}$, since $\langle g \rangle$. There is an inverse for all elements since the elements
	are $g^{\pm n}$. 
\end{proof}

\begin{thm}[3.29]
	Let $G$ be a group and $S$ be a subset of $G$. Then $\langle S \rangle$ is a subgroup of $G$. Moreover, if $H$ is a subgroup of $G$ and 
	$S\subset H$, then the subgroup $\langle S \rangle$ is a subgroup of $H$.
\end{thm}
\begin{proof}
	If $S$ is a single element, then $S$ is a group from Tm. 3.27. If $S$ contains multiple elements, than it is still a group because
	it has the identity, $e = a^n * a^{-n}$. All elements have an inverse since the elements are $a^{\pm n}$. It is also closed since all elements
	when operated upon still return an member of the group. Therefore, this is a group. This is also a subgroup since $S\subset H \subset G$. 
\end{proof}

\begin{ex}[3.30]
	Which subgroup is $\langle \{6, -8\}\rangle$? Which subgroup is $\langle \{5,-8\} \rangle$?
\end{ex}
\begin{solution}
	 $\langle \{6, -8\}\rangle$ is the subgroup $(2\mathbb{Z}, +)$. All elements are even number, since the addition and subtraction of
	 even numbers result in an even number, so it is not possible to get an odd number if you start with only even numbers.
	  $\langle \{5,-8\} \rangle$ is the subgroup $(\mathbb{Z}, +)$. $5+ - 8 = 3, 5-3=2, 3-2=1$. Since you now have one, you can get every integer
	  by either adding or subtracting one to the previous integer (the integer with the less absolute value). 
\end{solution}

\begin{thm}[3.33]
	Any subgroup of a cyclic group is cyclic. 
\end{thm}
\begin{proof}
	The subgroup of a cyclic group, since it is a group, contains the identity elements and the inverse for all members. Since the supergroup
	is cyclic, and the subgroup contains these specific elements, the cyclic property will be found in the subgroup. 
\end{proof}

\begin{thm}[3.34]
	The groups $D_n$ for $n\le 2$ are not cyclic. 
\end{thm}
\begin{proof}
	The groups for $n \le 2$ are not cyclic since there does not exist an element in $D_n$ that can generate itself, for $n\le2$. 
\end{proof}

\begin{thm}[3.36]
	Every finite group $G$ is finitely generated. 
\end{thm}
\begin{proof}
	If $G$ was not finitely generated, than $G$ would somehow not be able to contain all members of the set that generated it, which is 
	not possible. Therefore, $G$ was finitely generated. 
\end{proof}

\begin{thm}[3.37]
	The group $\{\mathbb{Q} \setminus \{0\}, \cdot\}$ is not finitely generated. 
\end{thm}
\begin{proof}
	This group is infinite, since the multiplication of fractions is infinite because it is not well-defined. Fractions can be represented in 
	many ways. Therefore, it is not finitely generated. 
\end{proof}

\end{document}
