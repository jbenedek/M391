\documentclass{article}
\usepackage{amsmath}
\usepackage{amsthm}
\usepackage{graphicx}
\usepackage{tikz}
\usepackage{amssymb}


% Options for amsthm
\newtheorem*{thm}{Theorem}
\newtheorem*{ex}{Exercise}
\newtheorem*{cor}{Corollary}
\newtheorem{lem}{Lemma}

\newenvironment{solution}
  {\begin{proof}[Solution]}
  {\renewcommand{\qedsymbol}{}\end{proof}}

\title{Reflection $2$}
\author{Jeremy Benedek}
\date{Feburary 19, 2016}

\begin{document}
\maketitle

\begin{description}
	\item[Part 1: Useful Mistakes] \hfill \\
		A lot of the mistakes I have made were mistakes I didn't originally notice, only until the grading process did I discover
		the mistakes. These mistakes help me realize the rigor required for proofs. Some specific mistakes I have made,
		are not properly doing if and only if proofs, I assume one side and prove the other, but I don't properly prove it by 
		additionally assuming the other side and proving the first side. This is an easy mistake to correct in the future, I 
		just need to finish the proof. Another mistake I make are in induction proofs, where I select
		the wrong base case. This too is an easy mistake to fix, since arguably in an induction proof the base case is the trivial step
		the real work comes from proving the claim works for the next item. I just need to double check the base case is indeed the correct
		base case. Another mistake I have made a few times is assuming one thing, and then proving my assumption. This goes against the 
		spirit of an implication, as I should assume something, and then prove something different. To remedy this, I need to be more mindful
		while solving proofs. 

	\item[Part 2: Types of Argument] \hfil \\
	  The first type of argument is proof by contradiction. The type of argument was used in proving Theorem 2.28. This argument works well
	  when dealing with a claim that assumes something is true, and as a result something else must also be true. Specifically, this works well when
	  the assumption deals with a property of a Graph.
	An outline of this proof would be: \\ 
	\begin{proof} Assume towards contradiction the negation of some part (or all) of the claim. By unpacking definitions, find a contradiction.
	  Since there is a contradiction, our assumption must be false. So we have the negation of assumption, which was the negation of the claim. 
	Therefore, we end with the claim, by using double negation. \end{proof}

	The second type of argument is proof by induction. This type of argument was used in proving Theorem 2.51. This argument works well when dealing
	with claims that involve manipulating a previous number/graph/etc. with the results still following some property. Summations are a specific
	example of this. An outline of this proof would be: \\
	\begin{proof} Prove the claim holds true for the base case. Now assume the claim works for the previous iteration and show the claim works
	  for the next iteration (so assume it works for $k$ and prove it works for $k + 1$). This assumption (the induction hypothesis) will
	mostly likely be used for a substitution during the process of proving it works for the next iteration. \end{proof}

	The third type of argument is proof by contrapositive. This type of argument was used in proving Theorem 2.31. This argument works well
	with conditionals. This argument style is an alternative to proof by implication. An outline for this proof would be: \\
	\begin{proof} If the claim is: If P, then Q, you would start by supposing the negation of Q. By using some proof techniques, you would need
	  to show that the negation of P is true (so P is false). 
	\end{proof}

	\item[Part 3: Three Polished Proofs] \hfil \\


	 	 \begin{cor}[2.58]
	    	If $G$ is a tree with $n$ vertices, then $G$ has $n - 1$ edges.
		      \end{cor}
		\begin{proof}
			A tree is a connected, planar graph, by Theorem 2.46, so by Theorem 2.55, $|V| - |E| + |F| = 2$. Since $G$ has $n$ vertices, 
			let $n$ = $V$, so
			$n - |E| + 1 = 2$. Since $G$ is a tree and therefore is planar, it only has 1 face, the unbounded face. $n - 1 = |E|$. Therefore the number of edges is $n - 1$.
		\end{proof}
		
		My proof for Corollary 2.58 has a interesting result since this theorem shows that there are many ways to prove a claim. We have proved this claim previously, and
		the fact that this claim can be proven using different arguments if important for proof writing, as there are many ways to 
		write a proof. 
	
		\begin{cor}[2.61]
		  	The graph $K_5$ is not planar.
		\end{cor}
		\begin{proof}
			Assume towards contradiction $K_5$ to be a planar graph. Since $K_5$ is a complete graph, it contains no loops and 
			has unique edges for each pair of distinct vertices. So by Theorem 2.59, If $|V| \geq 3$, then $|E| \leq 3|V| - 6.$
			$K_5$ has 5 vertices and 10 edges. $10 \nleq 3*5 - 6$ contradiction, so $K_5$ is not planar. 
		\end{proof}
	
		Corollary 2.61 has an important argument, that $K_5$ is planar. $K_5$ is a graph that we have been looking at since the
		start of the semester, and it important that we can now definitely proved that the 5-station problem is not possible. The ability
		to determine if a graph is planar in also important. 

		\begin{thm}[2.29]
		  	Let $G$ be a graph. Let $C$ be a subgraph of $G$ that consists of the vertices and edges that belong to a circuit in $G$. Then $deg_C(v)$ is 
			even for every vertex $v$ of $C$.
	       \end{thm}
	       \begin{proof}
			 Let $C$ be a subgraph of $G$ that consists of the vertices and edges that belong to a circuit in $G$.
			 Consider taking a walk through $C$. Since $C$ is a circuit, by definition a walk must end at the same vertex it began at, and
			 edges can not be repeated. So, for every time you walk into a vertex, 
		 	 you must walk away from the vertex on another edge, since you cannot repeat edges and must return at the starting vertex, so
			 the walk cannot end at that vertex, unless it is the first vertex in the walk. 
			 So for every passage through a vertex, there  must be a multiple of  
			 2 edges with endpoints at that vertex (one edge could be used for entry, and the other for exiting the vertex, and 
			 this process could be repeated $m$ times if the degree of the vertex is $2m$.
			 Therefore, for $n$ amount of passages through a vertex, 
			 there are $2n$ edges with endpoints at that vertex, so the degree of that vertex is $2n$, which is even since 2 times a 
			 integer is even.
		\end{proof}
						       
		My proof for Theorem 2.29 has 3 characteristics of good style in the exposition. First, I clearly define the notation and variables
		used in the proof, such as $C$ and $G$. Second, I provide definitions of terms, such as circuit. Lastly, my proof is concise and
		does not contains information that is not used. These are all examples of good style. 

\end{description}
\end{document}
