\documentclass{article}
\usepackage{amsmath}
\usepackage{amsthm}
\usepackage{graphicx}
\usepackage{tikz}
\usepackage{amssymb}

% Options for amsthm
\newtheorem*{thm}{Theorem}
\newtheorem*{ex}{Exercise}
\newtheorem*{cor}{Corollary}
\newtheorem{lem}{Lemma}
\newtheorem*{sch}{Scholium}

\newenvironment{solution}
  {\begin{proof}[Solution]}
  {\renewcommand{\qedsymbol}{}\end{proof}}

\title{Homework $5$}
\author{Jeremy Benedek}
\date{Feburary 14, 2016}

\begin{document}
\maketitle

\begin{thm}[2.51]
	Every connected graph can be created by starting with a single vertex and repeatedly adding one additional edge at a time to create increasingly larger connected subgraphs until the whole graph is created.
\end{thm}
\begin{proof}
	Use Strong induction on the number of edges. \\ Base: A connected graph with zero edges, so a single vertex. This graph is connected. Now assume a connected graph with at most $K$ edges can be built by repeatedly
	adding one additional edge at a time, for some $K \in \mathbb{N}$. Take this graph, and add an edge that has at least one endpoint already in the graph. This edge does not make the graph not connected since it has one
	endpoint in the graph. So the graph with $k+1$ edges is a connected graph.
\end{proof}

\begin{sch}[2.52]
	Let $G$ be a connected graph. Then $G$ can be constructed according to the previous theorem where all type (1) edges area added before any type (2) edge is added. If in addition $G$ is planar, this construction allows
	us to redraw $G$ using these techniques such that each partial drawing is also planar.
\end{sch}
\begin{proof}
	Start to construct $G$ by following 2.51, so start with a single vertex. Now add additional type 1 edges until no type 1 edges remain. This will result in a planar partial drawing since type 1 edges can be constructed
	so as not to cross. You now have all vertices of $G$ draw. The next step is to add in type 2 edges until no remain. If $G$ was planar, then you can add in each subsequent type 2 edge in such a manner that they do 
	not cross any other edge, maintain the planarity of the partial drawing.
\end{proof}

\begin{ex}[2.54]
	Draw a graph using the two procedures detailed in the Constructing Connected Graphs Theorem. Create a chart that includes the number of vertices, number of edges, and number of faces at each stage. Do you notice any patterns?
\end{ex}
\begin{solution} See the attached page at the end to see the graph. Here is the chart.\\
	Step \quad V \quad E \quad F \\
	1 \quad \quad  1 \quad \quad 0 \quad \quad 1 \\
	2 \quad \quad  2 \quad \quad 1 \quad \quad 1 \\
	3 \quad \quad  3 \quad \quad 2 \quad \quad 1 \\
	4 \quad \quad  3 \quad \quad 3 \quad \quad 2 \\
	5 \quad \quad  3 \quad \quad 4 \quad \quad 3 \\

	I notice that adding type 2 vertices will add 1 face, while add type 1 vertices doesn't add a face. Type one vertices also add a vertex, but type two vertices does not add a vertex. 
\end{solution}

\begin{thm}[2.55]
	For a connected graph $G$ drawn in the plane, $|V| - |E| + |F| = 2$.
\end{thm}
\begin{proof}
	Induct on the number of edges in a connected graph $G$. \\Base: there are zero edges. $E = 0$, $V = 1$, $F = 1$. $ 1 - 0 + 1 = 2$ \\ Assume for any connected graph with $k$ number of edges, $|V| - |E| + |F| = 2$. 
	Following Theorem 2.51, there are 2 types of edges that could be added, edges that have one endpoint already in the graph, and edges that already have both endpoints in the graph. If you add an edge with only 1 endpoint
	already in the graph, then you add 1 vertex while adding 1 edge and maintaining the same number of faces since the new edge will not enclose any portion of the graph. $|V + 1| - |E + 1| + |F| = |V| - |E| + |F| = 2$.
	If you add the other type of edge, than you maintain the number of vertices, add one edge, and add one face since the new edge will be enclosing a region. $|V| - |E + 1| + |F + 1| = |V| - |E| + |F| = 2$. 
\end{proof}

\begin{cor}[2.56]
	For a graph $G$ drawn in the plane with $n$ components, $|V| - |E| + |F| = n + 1$.
\end{cor}
\begin{proof}
	Induct on the number of components in connected graph $G$. \\ Base: there is one connected component. $|V| - |E| + |F| = 1 + 1$. This follows Theorem 2.55.\\ Assume for any connected graph with $k$ number of components.
	$|V| - |E| + |F| = n + 1$. 
	Now add a component to this connected graph. The starting component will have $n$ vertices, $e$ edges, and $f + 1$ faces, $f$ is the number of faces the component has, and then we add 1 to include the unbounded face to 
	the total number of faces, and 1 component. Now add another component, which will have $m$ vertices, $d$ edges, and $a + 1$ faces, and 1 component. The new graph will have $n + m$ vertices, $e + d$ edges, 
	and $f + a + 2$ vertices, and 2 components. Let $V$ be $n + m$, $E$ be $e + d$, and $F$ be $f + a + 2$. So, $|V| - |E| + |F| = n + 2$. We must have add 2 to the right side of the equation from Theorem 2.55, since we are
	counting 2 additional faces. However, this is an over calculation of faces since the is accounting for the unbounded face twice, but it is shared by the components, so we need to only account for the unbounded face once. So,
	$|V| - |E| + |F| = n + 1$.
\end{proof}

\begin{cor}[2.58]
	If $G$ is a tree with $n$ vertices, then $G$ has $n - 1$ edges.
\end{cor}
\begin{proof}
	A tree is a connected, planar graph, by Theorem 2.46, so by Theorem 2.55, $|V| - |E| + |F| = 2$. Since $G$ has $n$ vertices, let $n$ = $V$, so
	$|n| - |E| + 1 = 2$. Since $G$ is planar, it only has 1 face, the unbounded face. $|n| - 1 = |E|$. Therefore the number of edges is $n - 1$.
\end{proof}

\begin{cor}[2.61]
	The graph $K_5$ is not planar.
\end{cor}
\begin{proof}
	Assume towards contradiction $K_5$ to be a planar graph. Since $K_5$ is a complete graph, it contains no loops and has unique edges for each pair of distinct vertices. So by Theorem 2.59, If $|V| \geq 3$, then $|E| \leq 3|V| - 6.$
	$K_5$ has 5 vertices and 10 edges. $10 \nleq 3*5 - 6$ contradiction, so $K_5$ is not planar. 
\end{proof}

\begin{thm}[2.63]
	Any planar graph with no loops or multiple edges has a vertex of degree at most 5.
\end{thm}
\begin{proof}
	Assume towards contradiction that a planar graph, $G$, has all vertices with degree of 6. This graph must contains, per Theorem 2.51 either type 1 or type 2 edges. Since $G$ is of finite size, one cannot
	continue to add an infinite number of type 1 edges, so type 2 edges must be added in in order for all the vertices to have a degree of 6, however adding in type 2 edges encloses regions, adding faces. After enough
	edges, one must cross edges, making the graph not planar. So, $G$ must have a least one vertex with degree $\leq$ 5 that way type 1 edges could be added, since they will not make a graph to be not planar.  
\end{proof}

\begin{ex}[2.64]
	Theorem $2.63$ requires all of its hypotheses, of which there are three. For each hypothesis, find a counterexample to the theorem if that hypothesis were removed.
\end{ex}
\begin{solution}
	Without the no loop hypotheses, consider graph $G$ such that $G = (V,E)$, $V=\{A,B,C,D\}$, \\ 
	$E = \{ \{A,A\}, \{A,A\}, \{A,B\}, \{B,B\}, \{B,B\}, \{B,C\}, \{C,C\}, \{C,C\}, \{C,D\}, \{D,D\}, \{D,D\}, \{D,A\} \}$. 
	This is a planar graph, but all of the edges have a degree of 6. \\
	With the no multiple edges hypotheses, consider graph $G_1$ such that $G_1 = (V_1,E_1)$. $V_1=\{A,B,C\}$, $E_1=\{ \{A,B\} \{A,B\}, \{A,B\}, \{B,C\}, \{B,C\}, \{B,C\}, \{C,A\}, \{C,A\}, \{C,A\} \}$. This is a planar graph,
	but all of the edges have a degree of 6. \\
	With the hypothesis that the graph be planar, consider the complete graph of 7, $K_7$. The degree of every vertex is 6, and according to Theorem 2.59, is not planar. 
\end{solution}

\begin{thm}[2.65]
	The graph $K_{3,3}$ is not planar.
\end{thm}
\begin{proof}
	Assume towards contradiction that $K_{3,3}$ is planar. Since the graph is planar, it must follow Theorem 2.55. So, $|V| - |E| + |F| = 2$. The graph has 6 vertices, 9 edges, and 13 faces. As we can see $10 \ne 2$, so we have 
	a contradiction. So, $K_{3,3}$ is not planar.
\end{proof}


\end{document}
