\documentclass{article}
\usepackage{amsmath}
\usepackage{amsthm}
\usepackage{graphicx}
\usepackage{tikz}
\usepackage{amssymb}


% Options for amsthm
\newtheorem*{thm}{Theorem}
\newtheorem*{ex}{Exercise}
\newtheorem*{cor}{Corollary}
\newtheorem{lem}{Lemma}

\newenvironment{solution}
  {\begin{proof}[Solution]}
  {\renewcommand{\qedsymbol}{}\end{proof}}

\title{Reflection $2$}
\author{Jeremy Benedek}
\date{FJanuary 19, 2016}

\begin{document}
\maketitle

\begin{description}
	\item[Part 1: Useful Mistakes] \hfill \\
		A lot of the mistakes I have made were mistakes I didn't originally notice, only until the grading process did I discover
		the mistakes. These mistakes help me realize the rigor required for proofs. Some specific mistakes I have made,
		are not properly doing if and only if proofs, I assume one side and prove the other, but I don't properly prove it by 
		additionally assuming the other side and proving the first side. This is an easy mistake to correct in the future, I 
		just need to finish the proof. Another mistake I make are in induction proofs, where I select
		the wrong base case. This too is an easy mistake to fix, since arguably in an induction proof the base case is the trivial step
		the real work comes from proving the claim works for the next item. I just need to double check the base case is indeed the correct
		base case. Another mistake I have made a few times is assuming one thing, and then proving my assumption. This goes against the 
		spirit of an implication, as I should assume something, and then prove something different. To remedy this, I need to be more mindful
		while solving proofs. 

	\item[Part 2: Types of Argument] \hfil \\
	
	\item[Part 3: Three Polished Proofs] \hfil \\
		Theorem 2.10 is a proof who's result is mathematically important. This idea of total degree being even does came up a some furthur exercises. Additionally, Now that I know a Euler Circuit requires all verticies to be of even, positive degree, this has made me wonder what is so special with even numbers, because we are seeing 2 interesting problems (what is a graph, and what is a Euler circuit) hinge on even numbers.
		
		\begin{thm}2.10     The total degree of any graph is even.\end{thm}
 		 \begin{proof}  A graph is a pair of sets, one set contains all vertices in the graph, and the other is a set of all edges in the graph.
       				The degree of a vertex is the number of times a vertex is an endpoint for any edge in the graph. 
				The total degree of a graph is the sum of the degree of all vertices in the graph. 
				An edge contains 2 endpoints. 
				2 times any number is even, so 2 times the number of edges is even 
				The total degree of the graph is 2 times the number of edges, since each has 2 endpoints, and each endpoint is a vertex. 
       				Therefore, the toal degree is even.  \end{proof}

		Theorem 2.25 has a interesting and important argument, the idea of an equivalance relation and various proporties of relations.	
		Personally, I have previous experience with relations, and I like doing work involving them, so this is part of the reason the argument
		is interesting. 

		\begin{thm}[2.25]
		          Let $G$ be a graph with vertices $u$, $v$, and $w$.
        	  \begin{enumerate}
        	    \item The vertex $v$ is connected to itself.
       		    \item If $u$ is connected to $v$ and $v$ is connected to $w$, then $u$ is connected to $w$.
        	    \item If $v$ is connected to $w$, then $w$ is connected to $v$.
         	 \end{enumerate}
 		 \end{thm}
 		  \begin{proof}
        		$v$ is connected to itself if there is a walk from $v$ to $v$. By defintion 2.6, $W: v$ is a walk. So $v$ is connected to itself. \qed
         		 \\
         		Assume there is a walk between $u$ and $v$ and another walk between $v$ and $w$. 
			So, $W: v_1, e_1, ... v_j, e_j, v_k, e_k, v_l, e_l$, where $v_j$     is vertex u, 
			$v_k$ is vertex v, $v_l$ is vertex w. $W$ contains a walk between $u$ and $v$ and a 
			walk between $v$ and $w$ and finally, a walk between $u    $ and $w$,  so $u$ is connected to $w$. \qed
         		 \\
         		Assume $v$ is connected to $w$, so there is a walk between $v$ and $w$. 
			Since a walk is a finite set of adjacent vertices and edges, being adjacent does not depend on order, 
			just that they are next to each other in some direction. For an endpoint, the order of the vertices do not matter.
			So, if there is a walk between $v$ and $w$, then there is a walk between $w$ and $v$. Therefore, $w$ is connected to $v$, 
			since $v$ is connected to $w$.
 		\end{proof}
	
		My proof for Corollary 2.11 has good style. This Corollary was proven using proof by contradiction, definitions were unpacked
		(especially the definitions of even and odd), the arithmetic I do is clean and easy to follow. All three characteristics are 
		characteristics of good style.
	
		\begin{thm}Corollary 2.11     Let $G$ be a graph, then the number of vertices in $G$ with odd degree is even. \end{thm}
 		\begin{proof}    According to Theorem 2.10,  the total degree of $G$ is even 
 			Assume towards contradiction that $G$ has an odd amount verticies with odd degree
  			An even number can be represented as $2n$ where $n \in \mathbb{Z}$ an odd number can be represented as $2n + 1$ where 
			$n \in \mathbb{Z}$
			The total degree of $G$ is the sum of the degree of all vertices, both with even degree and odd degree.
			A vertex with even degree has degree $2n$. A vertex with odd degree has degree $2n + 1$  
  			So the total degree of $G$ is $2n\cdot k + (2n+1)\cdot j$ where $k$ is the amount of verticies with even degree and $j$
			is the amount of vertices with odd degree.
  			Using proporties of even and odd numbers, the addition of 2 even numbers is even, and the addition of an even number 
			and an odd number is odd. Additionally, an even number times an even number is even, 
			and an even number times an odd number is odd.
			Since there is an odd number of vertices with odd degree $j$ is odd, so $(2n+1)\cdot j$ is odd. 
			This number, added to $2n\cdot k$, which is even, will result in an odd number. Therefore, the 
			total degree of $G$ is odd, which contradicts Theorem 2.10, so $G$ must have an even amount of verticies with odd degree.
  		\end{proof}


\end{description}
\end{document}
