\documentclass{article}
\usepackage{amsmath}
\usepackage{amsthm}
\usepackage{graphicx}
\usepackage{tikz}
\usepackage{amssymb}


% Options for amsthm
\newtheorem*{thm}{Theorem}
\newtheorem*{ex}{Exercise}
\newtheorem*{cor}{Corollary}
\newtheorem{lem}{Lemma}

\newenvironment{solution}
  {\begin{proof}[Solution]}
  {\renewcommand{\qedsymbol}{}\end{proof}}

\title{Reflection $1$}
\author{Jeremy Benedek}
\date{January 29, 2016}

\begin{document}
\maketitle

\begin{description}
	\item[Part 1: Proof Construction] \hfill \\
		When I am stuck on a math problem or proof, I like to work through examples of the argument to fully convince myself that
		what I am proving is true. If that doesn't work, then I like to think about the goals for both the entire argument 
		and whatever step I am stuck on. Once I have a goal to
		work through, the problem is more manageable since I can see what I need in order to move on. Goals also help, because if I am stuck
		on one part, I know what I am supposed to get, but maybe not the actually manner to get it. If that is the case, I can pretend
		that I did get it, and finish the proof. I can then later go back and work on the one part I was stuck on. 
		If that doesn't work, then I think about what is stumping me and walk away and take a break, so when I come back to the problem, I 
		hopefully have an idea of where to start.
		Finally, if none of the previous tips work, I double check my work looking for errors
		that would interfere with  me solving it, and if I don't see any, I restart the problem.  
	\item[Part 2: Presentations] \hfil \\
		To increase the clarity of a presentation, a presenter should write big and legibly so all students can see it. Additionally, 
		a presenter should step back from the board occasionally to allow students to read what they just wrote, and also to explain 
		the step they just wrote. This process of stepping back also allows students to ask clarifing questions and maybe even spot
		little notation errors or typos that the presenter didn't realize. The presenter should definetly explain their notation they used,
		and this would be a good time to do so.  Finally, a presenter should be willing to answer questions 
		from the audience.  
	
	\item[Part 3: Three Polished Proofs] \hfil \\
		Theorem 2.10 is a proof who's result is mathematically important. This idea of total degree being even does came up a some furthur exercises. Additionally, Now that I know a Euler Circuit requires all verticies to be of even, positive degree, this has made me wonder what is so special with even numbers, because we are seeing 2 interesting problems (what is a graph, and what is a Euler circuit) hinge on even numbers.
		
		\begin{thm}2.10     The total degree of any graph is even.\end{thm}
 		 \begin{proof}  A graph is a pair of sets, one set contains all vertices in the graph, and the other is a set of all edges in the graph.
       				The degree of a vertex is the number of times a vertex is an endpoint for any edge in the graph. 
				The total degree of a graph is the sum of the degree of all vertices in the graph. 
				An edge contains 2 endpoints. 
				2 times any number is even, so 2 times the number of edges is even 
				The total degree of the graph is 2 times the number of edges, since each has 2 endpoints, and each endpoint is a vertex. 
       				Therefore, the toal degree is even.  \end{proof}

		Theorem 2.25 has a interesting and important argument, the idea of an equivalance relation and various proporties of relations.	
		Personally, I have previous experience with relations, and I like doing work involving them, so this is part of the reason the argument
		is interesting. 

		\begin{thm}[2.25]
		          Let $G$ be a graph with vertices $u$, $v$, and $w$.
        	  \begin{enumerate}
        	    \item The vertex $v$ is connected to itself.
       		    \item If $u$ is connected to $v$ and $v$ is connected to $w$, then $u$ is connected to $w$.
        	    \item If $v$ is connected to $w$, then $w$ is connected to $v$.
         	 \end{enumerate}
 		 \end{thm}
 		  \begin{proof}
        		$v$ is connected to itself if there is a walk from $v$ to $v$. By defintion 2.6, $W: v$ is a walk. So $v$ is connected to itself. \qed
         		 \\
         		Assume there is a walk between $u$ and $v$ and another walk between $v$ and $w$. 
			So, $W: v_1, e_1, ... v_j, e_j, v_k, e_k, v_l, e_l$, where $v_j$     is vertex u, 
			$v_k$ is vertex v, $v_l$ is vertex w. $W$ contains a walk between $u$ and $v$ and a 
			walk between $v$ and $w$ and finally, a walk between $u    $ and $w$,  so $u$ is connected to $w$. \qed
         		 \\
         		Assume $v$ is connected to $w$, so there is a walk between $v$ and $w$. 
			Since a walk is a finite set of adjacent vertices and edges, being adjacent does not depend on order, 
			just that they are next to each other in some direction. For an endpoint, the order of the vertices do not matter.
			So, if there is a walk between $v$ and $w$, then there is a walk between $w$ and $v$. Therefore, $w$ is connected to $v$, 
			since $v$ is connected to $w$.
 		\end{proof}
	
		My proof for Corollary 2.11 has good style. This Corollary was proven using proof by contradiction, definitions were unpacked
		(especially the definitions of even and odd), the arithmetic I do is clean and easy to follow. All three characteristics are 
		characteristics of good style.
	
		\begin{thm}Corollary 2.11     Let $G$ be a graph, then the number of vertices in $G$ with odd degree is even. \end{thm}
 		\begin{proof}    According to Theorem 2.10,  the total degree of $G$ is even 
 			Assume towards contradiction that $G$ has an odd amount verticies with odd degree
  			An even number can be represented as $2n$ where $n \in \mathbb{Z}$ an odd number can be represented as $2n + 1$ where 
			$n \in \mathbb{Z}$
			The total degree of $G$ is the sum of the degree of all vertices, both with even degree and odd degree.
			A vertex with even degree has degree $2n$. A vertex with odd degree has degree $2n + 1$  
  			So the total degree of $G$ is $2n\cdot k + (2n+1)\cdot j$ where $k$ is the amount of verticies with even degree and $j$
			is the amount of vertices with odd degree.
  			Using proporties of even and odd numbers, the addition of 2 even numbers is even, and the addition of an even number 
			and an odd number is odd. Additionally, an even number times an even number is even, 
			and an even number times an odd number is odd.
			Since there is an odd number of vertices with odd degree $j$ is odd, so $(2n+1)\cdot j$ is odd. 
			This number, added to $2n\cdot k$, which is even, will result in an odd number. Therefore, the 
			total degree of $G$ is odd, which contradicts Theorem 2.10, so $G$ must have an even amount of verticies with odd degree.
  		\end{proof}


\end{description}
\end{document}
