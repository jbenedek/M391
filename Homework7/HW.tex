\documentclass{article}
\usepackage{amsmath}
\usepackage{amsthm}
\usepackage{graphicx}
\usepackage{tikz}
\usepackage{amsfonts}
% Options for amsthm
\newtheorem*{thm}{Theorem}
\newtheorem*{ex}{Exercise}
\newtheorem*{cor}{Corollary}
\newtheorem{lem}{Lemma}

\newenvironment{solution}
  {\begin{proof}[Solution]}
  {\renewcommand{\qedsymbol}{}\end{proof}}

\title{Homework $7$}
\author{Jeremy Benedek}
\date{March 6, 2016}

\begin{document}
\maketitle

\begin{ex}[3.1]
	Show that there are exactly six transformations of the equilateral triangle. List it's inverse. 
\end{ex}
\begin{solution}
    Write your solutions here.
\end{solution}

\begin{ex}[3.4]
	Find some examples of two transformations of an equilateral triangle where composing the transformations in one order gives a different result from doing them in the other order.
\end{ex}
\begin{solution}
    Write your solutions here.
\end{solution}

\begin{ex}[3.5]
	Show that the following operations (*) are or are not closed, well-defined binary operations on the given sets. 
	\begin{enumerate}
	  \item The interval $[0,1]$ with $a*b = min\{a,b\}$
	  \item $\mathbb{R}$ with $a*b = a/b$
	  \item $\mathbb{Z}$ with $a*b = a^2 + b^2$
	  \item $\mathbb{Q}$ with $a*b = \frac{Numerator of b} {Decominator of b}$
	  \item $\mathbb{N}$ with $a*b = a-b$
	\end{enumerate}
\end{ex}
\begin{solution}
    Write your solutions here.
\end{solution}

\begin{thm}[3.6]
	Let $G$ be a group. There is a unique identity element in $G$. In other words, there is only one element in $G$, $e$, such that $g*e = g = e*g$ for all $g$ in $G$.
\end{thm}
\begin{proof}
    Write your proof here.
\end{proof}

\begin{thm}[3.7]
	Let $G$ be a group and let $a,x,y \in G$. Then $a*x = a*y$ if and only if $x=y$. Similarly, $x*a = y*a$ if and only if $x=y$.
\end{thm}
\begin{proof}
    Write your proof here.
\end{proof}

\begin{ex}[3.8]
	Show that the Cancellation law fails for $(\mathbb{R}, \cdot)$, thus confirming it is not a group.
\end{ex}
\begin{solution}
    Write your solutions here.
\end{solution}

\begin{cor}[3.9]
	Let $G$ be a group. Then each element $g$ in $G$ has a unique inverse in $G$. In other words, for a fixed $g$, there is only one element, $h$, such tat $g*h = e = h*g$. 
\end{cor}
\begin{proof}
    Write your proof here.
\end{proof}

\begin{thm}[3.10]
	Let G be a group with elements $g$ and $h$. If $g*h = e$, then $h*g=e$.
\end{thm}
\begin{proof}
    Write your proof here.
\end{proof}

\begin{thm}[3.11]
	Let $G$ be a group and $g\in G$. Then $(g^{-1})^{-1} = g$. 
\end{thm}
\begin{proof}
    Write your proof here.
\end{proof}

\begin{thm}[3.13]
	For every natural number n, the set $C_n$ with n-cyclic addition, ($C_n$, $\oplus_n$) is a group.
\end{thm}
\begin{proof}
    Write your proof here.
\end{proof}



\end{document}

