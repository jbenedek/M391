\documentclass{article}
\usepackage{amsmath}
\usepackage{amsthm}
\usepackage{graphicx}
\usepackage{tikz}
\usepackage{amsfonts}
\usepackage{amssymb}

% Options for amsthm
\newtheorem*{thm}{Theorem}
\newtheorem*{ex}{Exercise}
\newtheorem*{cor}{Corollary}
\newtheorem{lem}{Lemma}
\newtheorem{sch}{Scholium}


\newenvironment{solution}
  {\begin{proof}[Solution]}
  {\renewcommand{\qedsymbol}{}\end{proof}}

\title{Homework $10$}
\author{Jeremy Benedek}
\date{April 3, 2016}

\begin{document}
\maketitle

\begin{ex}[3.39]
	Compute the order of each element $T\in D_4$.
\end{ex}
\begin{solution}
	$\mathcal{O}(R_0) = 1$ \\
	$\mathcal{O}(R_{90}) = 4$ \\ 
	$\mathcal{O}(R_{180}) = 2$ \\
	$\mathcal{O}(R_{270}) = 4$ \\
	$\mathcal{O}(F_v) = 2$ \\ 
	$\mathcal{O}(F_h) = 2$ \\
	$\mathcal{O}(F_{D_1}) = 2$ \\
	$\mathcal{O}(F_{D_2}) = 2$
\end{solution}

\begin{thm}[3.42]
	If $G$ is a cyclic group, then $G$ is abelian.
\end{thm}
\begin{proof}
	Let $G$ be generated by $a$ and let $g$ and $h$ be elements of $G$. $g=a^m$ and $h=a^n$, $m,n \in \mathbb{Z}$. 
	$g*h = h*g$ \\ So, $a^n * a^m = a^m * a^n$ \\ $a^{n+m} = a^{m+n}$ Since addition is cumulative, this cyclic group is abelian. 
\end{proof}

\begin{ex}[3.44]
	\begin{enumerate}
	  \item Give an example of an infinite group that is abelian but not cyclic.
	  \item Give an example of a finite group that is abelian but not cyclic.
	\end{enumerate}
\end{ex}
\begin{solution}
	\begin{enumerate}
	  \item $\{\mathbb{R}\setminus \{0\}, \cdot\}$
      \item The group of $\mathbb{Z}_4 \cdot \mathbb{Z}_2$
	\end{enumerate}

\end{solution}


\begin{ex}[3.46]
	Give examples of Groups $G$ in which \begin{enumerate}
	  \item $Z(G) = \{e\}$
	  \item $Z(G) = G$
	  \item $\{e\} \subsetneq Z(G) \subsetneq G$
	    \end{enumerate}
\end{ex}
\begin{solution}
  \begin{enumerate}
    \item $Z(D_3) = \{R_)\}$
    \item $\{2\mathbb{Z}, +\}$
    \item $Z(D_4) = \{R_0, R_{180}\}$
  \end{enumerate}
\end{solution}


\begin{ex}[3.47]
	\begin{enumerate}
	  \item Consider $H = \{R_0, [flip across a horizontal line]\}$ a subgroup of $D_4$. Write out the left cosets of $H$. Also write out 
	    the right cosets of  $H$.
	  \item Consider $K=\{[0]_{12}, [3]_{12}, [6]_{12}, [9]_{12}\}$ a subgroup of $\mathbb{Z}_{12}$. Write out the left and right cosets.
	\end{enumerate}	
 \end{ex}
\begin{solution}
	\begin{enumerate}
	  \item The left cosets of: $R_0H = \{R_0, F_H\}$; $R_{90}H = \{R_{90}, D_2\}$; $R_{180}H = \{R_{180}, F_V\}$; $R_{270}H = \{R_{270}, D_1\}$;
	    $F_HH = \{F_H, R_0\}$; $F_VH = \{F_V, R_{180}\}$; $F_{D_1}H = \{F_{D_1}, R_{270}\}$; $F_{D_2}H = \{F_{D_2}, R_{90}\}$

	\end{enumerate}
	  
\end{solution}

\begin{lem}[3.48]
	Let $H$ be a subgroup of $G$ and let $g$ and $g\prime$ be elements of $G$. Then the cosets $gH$ and $g\prime H$ are either identical or disjoint.
\end{lem}
\begin{proof}
	Let $g, g\prime \in G$ and $h\in H$ $g*h = g\prime h$ Since $h=h$, for $gh$ to equal $g\prime h$, $g$ must equal $g\prime$. If $gh \neq g\prime h$
	$g\neq g\prime$
\end{proof}

\begin{thm}[3.49]
	Let $G$ be a finite group with subgroup $H$. Then $|H|$ divides $|G|$.
\end{thm}
\begin{proof}
    Write your proof here.
\end{proof}

\begin{sch}[3.51]
	Let $G$ be a finite group with a subgroup $H$. Then $[G : H] = |G|/|H|$.
\end{sch}
\begin{proof}
    Write your proof here.
\end{proof}

\begin{cor}[3.52]
  Let $G$ be a finite group with an element $g$. Then $\mathcal{O}(g)$ divides $|G|$. 
\end{cor}
\begin{proof}
    Write your proof here.
\end{proof}

\end{document}
