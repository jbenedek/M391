\documentclass{article}
\usepackage{amsmath}
\usepackage{amsthm}
\usepackage{graphicx}
\usepackage{tikz}
\usepackage{amssymb}
\usepackage{amsfonts}

% Options for amsthm
\newtheorem*{thm}{Theorem}
\newtheorem*{ex}{Exercise}
\newtheorem*{cor}{Corollary}
\newtheorem{lem}{Lemma}

\newenvironment{solution}
  {\begin{proof}[Solution]}
  {\renewcommand{\qedsymbol}{}\end{proof}}

\title{Reflection $4$}
\author{Jeremy Benedek}
\date{April 8, 2016}

\begin{document}
\maketitle

\begin{description}
	\item[Part 1: Proof Writing Steps] \hfill \\
		When I am presented a claim to prove, I first make sure I 
		to reference the definitions of terms used in the claim. I then think of
		possible previous theorem that may help, and look at when those proofs are applicable.  
		With this knowledge, I then develop a plan on how to solve the claim. This plan will also include goals, or things I should
		try and prove are true so I can use them in the future. The next step is proving these little side-claims from my plan are true, and
		thus can be used in my proof. The final step is
		finally formulating a well-written proof by including all of these items in one cohesive proof.
		\\
		I spend most of my time on making my plan, and then proving the small claims that I outlined in my plan, so steps 3 and 4 above. 
		The two steps that discussion with classmates help are thinking of previous theorems and when  you can use those, and then coming
		up with the plan. In both cases, my classmates may have a different perspective or may be remembering something that I am not. 

	\item[Part 2: The Language of Mathematics] \hfil \\
		The writing style used in mathematics and in the humanities are similar in that the writing must be clear and understandable to
		the reader and both writing styles often use some
		outside knowledge that is cited (in math, we use previous theorems and definitions, in the humanities, we use articles, books, and
		other sources). These writing styles differ in that the math writing will contains symbols and numbers, while the humanity writing 
		style uses almost exclusively words. Additionally, the grammar and flow of the sentences vary greatly between the two styles. In a 
		mathematical style, sentences are short and concise and all have the same structure.
		While in the humanities, the writing uses paragraphs and long elegant sentences
		and the structure of the sentence varies. 

	\item[Part 3: Three Polished Proofs] \hfil \\
		My proof for Thm. 3.24 has a result that is important because this proof shows us that the smallest group possible is the group
		containing just the identity element. 
	  \begin{thm}[3.24]
	      	Let $G$ be a group with identity element $e$. Then $\{e\}$ is a subgroup of $G$. 

		\end{thm}
	      \begin{proof}
	      $\{e\}$ is a group since it has an identity element, $e$, and all members have an inverse, in this case with only one element, $\{e\}$ has
	an inverse of $\{e\}$. The associativity and well-defined requirements of a group are inherited from the $G$, so it follows that $\{e\}$ also
	have these characteristics. This will be closed since it is the identity element by itself, so regardless of the operation, $e*e$ will
	have a result always in $\{e\}$. Therefore, this is a group, and since all members are in $G$, this is a subgroup of $G$.

	      \end{proof}

	      My proof for Thm. 3.25 has an argument that is important an interesting because the argument that a group is a subgroup of itself 
	      has further implications and affect my interpretation and understanding of a subgroup. 
	      This argument is interesting because it implies that every group will have at least one subgroup, itself. 
		\begin{thm}[3.25]
			Let $G$ be a group. Then $G$ is a subgroup of $G$. 

		\end{thm}
		\begin{proof}
		 Since $G$ is a group, $G$ is closed, well-defined, associative, has an identity element, and all members have an inverse. So it follows
	that $G$ is a group. Since $G$ is the same group as $G$, all members of $G$ exist in $G$, so $G$ is a subgroup of $G$.

		\end{proof}


		My proof for Thm. 3.27 has 3 characteristics of good style. First, my proof is concise and straight to the point.
		Second, I use relevant definitions by showing it is a group and thus follows all 5 parts of the definition of a group.
		Lastly, my notation is clear and I do not introduce an extraneous variables. 
		\begin{thm}[3.27]
		  	Let $G$ be a group and $g$ be an element of $G$. Then $\langle g \rangle$ is a subgroup of $G$. 

	      
		\end{thm}
	      \begin{proof}
	      All members of $\langle g \rangle$ are members of $G$ since $\langle g \rangle$ is formed by using the members of $G$. $\langle g \rangle$
	is also a group since it fits the definition of a group by having all 5 requirements; it is associative and well-defined since 
	that is inherited from the bigger group. The identity element 
	is present in this subgroup since $e = g^0$, since $\langle g \rangle$. 
	There is an inverse for all elements since the elements
	are $g^{\pm n}$. Lastly, this is closed as operations between members of the group will have the result also be in the group. So this is a 
	group. 
	      \end{proof}

\end{description}
\end{document}
