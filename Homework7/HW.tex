\documentclass{article}
\usepackage{amsmath}
\usepackage{amsthm}
\usepackage{graphicx}
\usepackage{tikz}
\usepackage{amsfonts}
% Options for amsthm
\newtheorem*{thm}{Theorem}
\newtheorem*{ex}{Exercise}
\newtheorem*{cor}{Corollary}
\newtheorem{lem}{Lemma}

\newenvironment{solution}
  {\begin{proof}[Solution]}
  {\renewcommand{\qedsymbol}{}\end{proof}}

\title{Homework $7$}
\author{Jeremy Benedek}
\date{March 6, 2016}

\begin{document}
\maketitle

\begin{ex}[3.1]
	Show that there are exactly six transformations of the equilateral triangle. List it's inverse. 
\end{ex}
\begin{solution}
	The transformation are $R_0$, with inverse of $R_0$; $R_{120}$, with inverse of $R_{240}$; $R_{240}$, with inverse of $R_{120}$; $F_T$, with inverse of $F_T$; $F_L$, with inverse of $F_L$; and $F_R$, with inverse of $F_R$. 
	These are the only six because there are three vertices in a triangle, and only two distinct ways of manipulating the triangle (flip and rotation). So 3 times 2 is 6. 
\end{solution}

\begin{ex}[3.4]
	Find some examples of two transformations of an equilateral triangle where composing the transformations in one order gives a different result from doing them in the other order.
\end{ex}
\begin{solution}
  $R_{120} \circ F_L \neq F_L \circ R_{120}$
\end{solution}

\begin{ex}[3.5]
	Show that the following operations (*) are or are not closed, well-defined binary operations on the given sets. 
	\begin{enumerate}
	  \item The interval $[0,1]$ with $a*b = min\{a,b\}$
	  \item $\mathbb{R}$ with $a*b = a/b$
	  \item $\mathbb{Z}$ with $a*b = a^2 + b^2$
	  \item $\mathbb{Q}$ with $a*b = \frac{Numerator of b} {Decominator of b}$
	  \item $\mathbb{N}$ with $a*b = a-b$
	\end{enumerate}
\end{ex}
\begin{solution}
	\begin{enumerate}
	  \item This is closed and well defined. Since the result of the minimum operation cannot be out of the interval, and there is no ambiguity representing numbers within that interval.
	  \item Not closed, Let $a=6$, $b=0$. $\frac{6}{0} \notin \mathbb{R}$
	  \item This is closed and well defined. Since the result of the operation will be an integer, and there is no ambiguity in representing an integer. 
	  \item This is closed since the resulting fraction of the operation will be in $\mathbb{Q}$ since it is a fraction. This is not well defined since there is ambiguity in representing fractions. Let $a=\frac{1}{4} = \frac{7}{28}$
	    and $b=\frac{28}{7}$. So the result of the operation could be $\frac{1}{7} or \frac{7}{7}$ depending on what representation is chosen. 
	  \item Not closed, Since if $b>a$, then $a-b < 0$, which is not in $\mathbb{N}$

	\end{enumerate}
\end{solution}

\begin{thm}[3.6]
	Let $G$ be a group. There is a unique identity element in $G$. In other words, there is only one element in $G$, $e$, such that $g*e = g = e*g$ for all $g$ in $G$.
\end{thm}
\begin{proof}
	Let $e_1, e_2$ be identity elements in $G$ such that $g*e_1 = g = e_1 * g$ and $g*e_2 = g = e_2 *g$\\So $g*e_1 = e_2 *g$. So $e_1 = e_2$. There are equal, so there is only 1 unique identity.
\end{proof}

\begin{thm}[3.7]
	Let $G$ be a group and let $a,x,y \in G$. Then $a*x = a*y$ if and only if $x=y$. Similarly, $x*a = y*a$ if and only if $x=y$.
\end{thm}
\begin{proof}
	Let $a*x = a*y$ \\ $a{-1}*a*x = a{-1}* a*y$ \\ $e*x = e*y$ for some identity element $e$.  So $x=y$.
	\\ Let  $a*y = a*x$ \\ $a{-1}*a*y = a{-1}* a*x$ \\ $e*y = e*x$ for some identity element $e$. So $x=y$.\\
	Let $x=y$. There exists an operation such that $x*a = y*a$. Likewise, Let $x=y$, there exist an operation such that $a*x = a*y$. 
\end{proof}

\begin{ex}[3.8]
	Show that the Cancellation law fails for $(\mathbb{R}, \cdot)$, thus confirming it is not a group.
\end{ex}
\begin{solution}
	Let $a=0, x=3, y=5$. So $0\cdot3 = 0\cdot5$. $0=0$, but $5\neq3$
\end{solution}

\begin{cor}[3.9]
	Let $G$ be a group. Then each element $g$ in $G$ has a unique inverse in $G$. In other words, for a fixed $g$, there is only one element, $h$, such tat $g*h = e = h*g$. 
\end{cor}
\begin{proof}
	Let $g*h_1 = e = h_1 * g$ and  $g*h_2 = e = h_2 * g$ \\  $g*h_1 = g*h_2$, by Theorem 3.7, $h_1 = h_2$.
\end{proof}

\begin{thm}[3.10]
	Let G be a group with elements $g$ and $h$. If $g*h = e$, then $h*g=e$.
\end{thm}
\begin{proof}
	By definition, there is an identity element and inverse for all elements in $G$. \\ $g * h & h^{-1} = e * h^{-1}$ \\ $g * e = e & h^{-1}$ \\ $g = h^{-1}$. $h*g = h * h^{-1}$. So, $h*g=e$
\end{proof}

\begin{thm}[3.11]
	Let $G$ be a group and $g\in G$. Then $(g^{-1})^{-1} = g$. 
\end{thm}
\begin{proof}
	Let $h = g^{-1}$. So $h^{-1} = e*h^{-1} = g * h * h^{-1}$ based on the definition of an identity element $e$. By the associative property, $= g * h * h^{-1} = g * e = g$. 	
\end{proof}

\begin{thm}[3.13]
	For every natural number n, the set $C_n$ with n-cyclic addition, ($C_n$, $\oplus_n$) is a group.
\end{thm}
\begin{proof}
	The operation $\oplus_n$ is closed and well-defined since the function is defined in such a way that the result of the operation is less than $n$. That operation is also associative since it is addition, which is associative on the 
	natural numbers. There is an identity element for all elements, 12, and an inverse for all elements, 1. So by definition, this is a group.
\end{proof}



\end{document}

