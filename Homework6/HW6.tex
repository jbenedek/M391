\documentclass{article}
\usepackage{amsmath}
\usepackage{amsthm}
\usepackage{graphicx}
\usepackage{tikz}

% Options for amsthm
\newtheorem*{thm}{Theorem}
\newtheorem*{ex}{Exercise}
\newtheorem*{cor}{Corollary}
\newtheorem{lem}{Lemma}

\newenvironment{solution}
  {\begin{proof}[Solution]}
  {\renewcommand{\qedsymbol}{}\end{proof}}

\title{Homework $6$}
\author{Jeremy Benedek}
\date{Feburary 21, 2016}

\begin{document}
\maketitle

\begin{thm}[2.70]
	There are only five regular solids.
\end{thm}
\begin{proof}
    Write your proof here.
\end{proof}

\begin{ex}[2.72]
	How many automorphisms does $K_5$ have? How many automorphisms does $K_{3,3}$ have?
\end{ex}
\begin{solution}
    Write your solutions here.
\end{solution}

\begin{ex}[2.73]
	For each of the 5 regular planar graphs, find all auto-morphisms of the graph.
\end{ex}
\begin{solution}
    Write your solutions here.
\end{solution}

\begin{ex}[2.74]
	Devise a computer program that determines whether two graphs are isomorphic.
\end{ex}
\begin{solution}
    Write your solutions here.
\end{solution}

\begin{ex}[2.76.5]
	Find the dual graph of each of the platonic solids.
\end{ex}
\begin{solution}
    Write your solutions here.
\end{solution}

\begin{thm}[2.78]
	For any natural number, $n$, let $G = (V,E)$ be a graph with $|V| \leq n$ that has no loops. Then $G$ is n-colorable. 
\end{thm}
\begin{proof}
    Write your proof here.
\end{proof}

\begin{thm}[2.79]
	Let $G = (V,E)$ be a graph with no loops. Let $M = max\{deg(v)|v\in V\}$, which is the biggest degree of a vertex. Then $G$ is $(M+1)$-colorable.
\end{thm}
\begin{proof}
    Write your proof here.
\end{proof}

\begin{thm}[2.80]
	Consider a graph, $G$, that is built from a subgraph, $H$, by adding one new vertex, $v$, and new edges that connect the new vertex to vertices in $H$. If the subgraph $H$ has a 5-coloring such that the new vertex, $v$, is 
	not adjacent to vertices of all five colors, then $G$ is 5-colorable.
\end{thm}
\begin{proof}
    Write your proof here.
\end{proof}

\begin{thm}[2.82]
	Any planar graph with no loops is 6-colorable.
\end{thm}
\begin{proof}
	Following Theorem 2.63, any planar graph with no loops has a vertex of degree at most 5. Using Theorem 2.79,  graph $G$ is $m+1$ colorable, with $m$ being the biggest degree of a vertex.
	Since 5 is the largest degree of a vertex in a planar graph, it is 5+1, or 6-colorable.
\end{proof}

\end{document}
